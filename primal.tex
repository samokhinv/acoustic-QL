\documentclass[a4paper,14pt]{article}


%%% Работа с русским языком
\usepackage[T2A]{fontenc}			% кодировка
\usepackage[utf8]{inputenc}	
\usepackage[english,russian]{babel}   %% загружает пакет многоязыковой вёрстки
\usepackage{indentfirst}
\usepackage{algorithm}
\usepackage[noend]{algpseudocode}
\usepackage{dsfont}
\renewcommand{\P}{\ensuremath{\mathds{P}}}
\renewcommand{\epsilon}{\ensuremath{\varepsilon}}
\renewcommand{\phi}{\ensuremath{\varphi}}
\renewcommand{\kappa}{\ensuremath{\varkappa}}
\renewcommand{\le}{\ensuremath{\leqslant}}
\renewcommand{\leq}{\ensuremath{\leqslant}}
\renewcommand{\ge}{\ensuremath{\geqslant}}
\renewcommand{\geq}{\ensuremath{\geqslant}}
\renewcommand{\emptyset}{\varnothing}
\newcommand{\ra}{\rightarrow}
\newcommand{\ol}[1]{\overline{#1}}
%%% Дополнительная работа с математикой
\usepackage{amsmath,amsfonts,amssymb,amsthm,mathtools} % AMS
\usepackage{icomma} % "Умная" запятая: $0,2$ --- число, $0, 2$ --- перечисление

%% Номера формул
%\mathtoolsset{showonlyrefs=true} % Показывать номера только у тех формул, на которые есть \eqref{} в тексте.
%\usepackage{leqno} % Нумерация формул слева

%% Свои команды
\DeclareMathOperator{\sgn}{\mathop{sgn}}

%% Перенос знаков в формулах (по Львовскому)
\newcommand*{\hm}[1]{#1\nobreak\discretionary{}
{\hbox{$\mathsurround=0pt #1$}}{}}

%%% Работа с картинками
\usepackage{tikz}
\usetikzlibrary{automata, positioning}
\usepackage{graphicx}  % Для вставки рисунков
\setlength\fboxsep{3pt} % Отступ рамки \fbox{} от рисунка
\setlength\fboxrule{1pt} % Толщина линий рамки \fbox{}
\usepackage{wrapfig} % Обтекание рисунков текстом



%%% Работа с таблицами
\usepackage{array,tabularx,tabulary,booktabs} % Дополнительная работа с таблицами
\usepackage{longtable}  % Длинные таблицы
\usepackage{multirow} % Слияние строк в таблице

%%% Теоремы
\theoremstyle{plain} % Это стиль по умолчанию, его можно не переопределять.
\newtheorem{theorem}{Теорема}[section]
\newtheorem{proposition}[theorem]{Утверждение}
\newtheorem{problem}{Задача}[section]

 
\theoremstyle{definition} % "Определение"
\newtheorem{lemma}{Лемма}[section]
\newtheorem{conclusion}{Следствие}[theorem]
 
\theoremstyle{remark} % "Примечание"
\newtheorem*{defin}{Определение}
\newtheorem*{answer}{Ответ}
\newtheorem*{solution}{Решение}
\newcommand{\E}{\ensuremath{\mathds{E}}}
\newcommand{\V}{\ensuremath{\mathds{V}}}
%%% Программирование
\usepackage{etoolbox} % логические операторы

%%% Страница
\usepackage{extsizes} % Возможность сделать 14-й шрифт
\usepackage{geometry} % Простой способ задавать поля
   \geometry{top=30mm}
   \geometry{bottom=40mm}
   \geometry{left=20mm}
   \geometry{right=20mm}
 %
%\usepackage{fancyhdr} % Колонтитулы
% 	\pagestyle{fancy}
 	%\renewcommand{\headrulewidth}{0pt}  % Толщина линейки, отчеркивающей верхний колонтитул
% 	\lfoot{Нижний левый}
% 	\rfoot{Нижний правый}
% 	\rhead{Верхний правый}
% 	\chead{Верхний в центре}
% 	\lhead{Верхний левый}
%	\cfoot{Нижний в центре} % По умолчанию здесь номер страницы

\usepackage{setspace} % Интерлиньяж
\onehalfspacing % Интерлиньяж 1.5
%\doublespacing % Интерлиньяж 2
\usepackage{caption}
\usepackage{lastpage} % Узнать, сколько всего страниц в документе.

\usepackage{soul} % Модификаторы начертания

\usepackage{hyperref}

\hypersetup{				% Гиперссылки
    unicode=true,           % русские буквы в раздела PDF
    pdftitle={Опты №2},   % Заголовок
    pdfauthor={Самохин 676},      % Автор
    colorlinks=true,       	% false: ссылки в рамках; true: цветные ссылки
    linkcolor=black,          % внутренние ссылки
    citecolor=black,        % на библиографию
    filecolor=magenta,      % на файлы
    urlcolor=blue           % на URL
}

\usepackage{csquotes} % Еще инструменты для ссылок
\usepackage{listings}
\usepackage{footnote}
\makesavenoteenv{tabular}

\usepackage{tikz}
\usepackage{tikz-3dplot}
\usepackage{multicol}
\renewcommand{\r}{\ensuremath{\overline{r}}}

\begin{document}
	\section{Прямая задача}
	Разделение акустического волнового поля на первичную и отраженную (фоновую и аномальную части)
	
	$P(\r, t)$ - давление в точке $\overline{r}$ в момент $t$\\
	Применяем преобразование Фурье:
	$$P(\r, t) = \dfrac{1}{2\pi}\int_{-\infty}^{+\infty}p(\r, \omega)e^{-i\omega t}d\omega$$
	$$p(\r, \omega) = \int_{-\infty}^{+\infty}P(\r, t)e^{i\omega t}dt$$
	
	Акустическое уравнение:
	$$\nabla^2p(\r, \omega) + \dfrac{\omega^2}{c^2(\r)}p(\r, \omega) = -f^e(\r, \omega),$$
	где 
	$$f^e = \int_{-\infty}^{+\infty}F^e(\r, t)e^{i\omega t}dt$$
	
	Введем граничные условия:
	$rp(\r, \omega)$ - ограничено и $$\displaystyle{\lim_{r\to\infty} r\left[\dfrac{\partial}{\partial r}p(\r, \omega) - i \dfrac{\omega}{c}p(\r, \omega)\right]} = 0$$
	
	Введем представление скорости в модели:
	$$\dfrac{1}{c^2(\r)} = \dfrac{1}{c^2_b}(1+a(\r)),$$
	где величина $a(\r)$ задана внутри аномальной области, где скорость отлична от фоновой.
	
	Считаем, что вызванная этим неоднородность локальна, то есть:
	$$\exists R: \forall \r:\ \|\r\| > R\; \rightarrow \c(\r) = c_b(\r)$$
	
	Для удобства введем понятие \underline{медленности}:
	$$s(\r) = \dfrac{1}{c(\r)},\; s_b(\r) = \dfrac{1}{c_b(\r)},$$	
	тогда 
	$$a(\r) = \dfrac{s^2(\r) - s^2_b(\r)}{S^2_b(\r)} = \dfrac{\Delta S^2(\r)}{S^2_b(\r)}$$
	
	\subsection{Разделение:}
	$$p(\r, \omega) = p^i(\r, \omega) + p^s(\r, \omega),$$
	где $p^i$ - первичное (initial) поле, $p^s$ - отраженное (secondary) поле.
	
	Уравнения после разделения примут вид:
	\begin{equation*}
		\begin{rcases}
		\nabla^2p^i(\r, \omega) + \dfrac{\omega^2}{c_b^2(\r)}p^i(\r, \omega) = - f^e(\r, \omega),\\
		\nabla^2p^s(\r, \omega) + \dfrac{\omega^2}{c_b^2(\r)}p^is\r, \omega) = - \omega^2\nabla s^2(\r)p(\r, \omega)
		\end{rcases}
		\text{удовл. гр.усл.}
	\end{equation*}
	$f^a(\r, \omega) = \omega^2\nabla s^2p(\r, \omega) $ - интенсивность аномального источника\\
	$\nabla^2p^s(\r, \omega) + \dfrac{\omega^2}{c^2_b(\r)}p^2(\r, \omega) = -f^a(\r, \omega)$
	
	Интегральное уравнение:
	$$p^i(\r_j, \omega) = \int\int_{V^\infty}\int f^e(\r, \omega)G^\omega(\r_j \mid \r, \omega)dv = G_\omega(f^e)$$
	Здесь под $\r_j$ понимается некоторая постоянная точка, в то время как по $\r$ ведется интегрирование.
	
	$G_\omega(f^e)$ - скалярный волновой оператор Грина,ёё
	$G^\omega(\r_j \mid \r,\omega)$ - функция Грина - хорошее решение уравнения для сигнала -- дельта-функции, \\
	$$\nabla^2 G^\omega(\r_j \mid \r, \omega) + \dfrac{\omega^2}{c^2_b(\r)}G^\omega(\r_j \mid \r, \omega) = - \delta(\r_j, -\r)$$
	Для отр. поля:
	$$p^s(\r, \omega) = \int\int_\mathcal{D}\int G^\omega(\r_j \mid \r, \omega) f^a(\r, \omega)d\r = G_\omega(f^a),$$
	то есть $p^s = \omega^2 G_\omega(\nabla s^2p)$.
	
	\subsection{Общее уравнение:}
	 $$p(\r_j, \omega) = \omega^2 G_\omega\left(\nabla s^2(\r)\ p(\r, \omega)\right) + p^i(\r, \omega),$$
	 где первое слагаемое соответствует отраженным волнам, а второе - первичным.
	 
	 $\r_j \in \mathcal{D} \Rightarrow$  инт. ур-е относительно волнового поля $p(\r,\omega)$
	 
	 \begin{theorem}[взаимности]
	 	Можно поменять роли точек "источник-приемник"
	 	\ местами, и результат не изменится:
	 	$$G^\omega(\r^{''} \mid \r^{'}, \omega) = G^\omega(\r^{'} \mid \r^{''}, \omega)$$
	 \end{theorem}
 
 	\subsection{Приближение Борна}
 	\begin{multline*}p^s(\r_j, \omega) = \omega^2G_\omega\left(\triangle s^2(\r)p(\r, \omega)\right) = \\ \omega^2 \int\int_\mathcal{D}\int G^\omega(\r_j \mid \r, \omega)\triangle s^2(\r)\left[p^i(\r,\omega) + p^s(\r,\omega)\right]dr
 	\end{multline*}
 	$p^B(\r,\omega)$ - приближение Борна: отбрасываем отраженное поле внутри области:
 	$$p^B(\r_j, \omega) = \omega^2 \int\int_\mathcal{D}\int G^\omega\left(\r_j \mid \r, \omega\right)\nabla s^2(\r)p^i(\r, \omega)dr$$
 	Теперь эта задача решается: $p^i$ можно вычислить, следовательно, можно посчитать и $p^B$.
 	
 	\subsection{Квазилинейное приближение}
 	$$p(\r, \omega) = p^i(\r, \omega) + p^s(\r, \omega)=\left[1 + \lambda(\r, \omega)\right]p^i(\r, \omega),\ \ \lambda(\r, \omega) = \dfrac{p^s(\r, \omega)}{p^i(\r, \omega)}$$
 	$$p^s(\r_j, \omega) = \omega^2G_\omega\left(\triangle s^2(\r)\left[1+\lambda(\r, \omega)\right]p^i(\r, \omega)\right)$$
 	Заметим, что при $\lambda  = 0$ получаем приближение Борна.
 	
 	При таком подходе основной проблемой становится нахождение $\lambda$, которая решается численным решением задачи минимизации
 	$$\|\lambda(\r_j, \omega)p^i(\r_j, \omega) - \omega^2G_\omega\left(\triangle s^2(\r)\left[1+ \lambda(\r, \omega)\right]p^i(\r, \omega)\right)\|_{\Omega, \mathcal{D}},$$
 	где под $\|\cdot\|_{\Omega, \mathcal{D}}$ подразумевается норма в $L_2$, т.е. 
 	$$\|p\|_{\Omega, \mathcal{D}} = \sqrt{\int_{\Omega}\ \int\int_\mathcal{D}\int (p(\r, u))^2drdu}$$
 	
 	\subsection{Квазианалитическое приближение}
 	Раскладываем $\lambda$ по Тейлору:
 	$$\lambda(\r, \omega) = \lambda(\r_j, \omega) + (\r - \r_j)\nabla\lambda(\r_j, \omega) + O(|\r - \r_j|^2)$$
 	Тогда
 	$$\lambda(\r, \omega)p^i(\r_j, \omega) \approx \omega^2 \lambda(\r_j,\omega)G_\omega\left[\triangle s^2p^i\right] + p^B = \lambda(\r_j, \omega)p^B(\r_j, \omega) + p^\beta(\r_j, \omega)$$
 	
 	$$\lambda(\r_j, \omega) = \dfrac{p^B(\r_j, \omega)}{p^i(\r_j, \omega) - p^\beta(\r_j, \omega)}, \ \ \r_j \in \mathcal{D}$$
 	
 	Таким образом,
 	$$p^s_{QA} = \omega^2 G_\omega\left[\triangle s^2(\r)\dfrac{p^i(\r, \omega)}{1-g(\r, \omega)}\right], \ \ g(\r, \omega) = \dfrac{p^B(\r, \omega)}{p^i(\r, \omega)}$$
 	
\end{document}